% LaTeX resume using res.cls
\documentclass[line,margin]{res}
\usepackage{hyperref}

\begin{document}

\name{James Bourbeau}
% \address used twice to have two lines of address
%\address{Wisconsin IceCube Particle Astrophysics Center}
%\address{222 West Washington Ave., Suite 500 Madison, WI 53703}

\begin{resume}

%============= Contact information ==================
\section{\sc Contact Information}

E-mail: james.bourbeau{\emph{@}}icecube.wisc.edu

Webpage: \href{https://jrbourbeau.github.io/}{https://jrbourbeau.github.io/}

GitHub: \href{https://github.com/jrbourbeau}{https://github.com/jrbourbeau}

LinkedIn: \href{https://www.linkedin.com/in/jrbourbeau/}{https://www.linkedin.com/in/jrbourbeau/}

%============= Education =========================
\section{\sc Education}
Ph.D. in Physics (in progress) \hfill 2013-Present \\
\textsc{University of Wisconsin--Madison}

M.S. in Physics \hfill 2017 \\
\textsc{University of Wisconsin--Madison}

B.S. in Physics with Honors \hfill 2013 \\
\textsc{University of Texas at Arlington}
%                \begin{itemize}  \itemsep -2pt % reduce space between items
%%                 \item Summa Cum Laude
%	        \item Minor: Mathematics
%	        \item Advisor: Andrew Brandt
%	        \item Thesis: {\it Detector Development for a High Precision Time of Flight Detector}
%	       \end{itemize}


%============= Publications =========================
\section{\sc Publications}
\begin{itemize}

	\item J. Bourbeau, P. Desiati, J.C. D�az V�lez, S. Westerhoff et al. (IceCube Collaboration),
		{\it Cosmic-Ray Anisotropy with Seven Years of Data from IceCube and IceTop},
		Proceedings of the 35th International Cosmic Ray Conference.
	\href{https://pos.sissa.it/301/474/pdf}{[proceedings]}.

	\item Y. Bai, J. Bourbeau, and T. Lin. {\it Dark Matter Searches with a Mono-$Z'$ Jet}. JHEP {\bf 1506}, 205 (2015).
	\href{http://arxiv.org/pdf/1504.01395.pdf}{[arXiv:1504.01395]}.

\end{itemize}

%============= Research Experience =========================
\section{\sc Research Experience}
	\textsc{IceCube Collaboration}  \hfill 2015-Present \\
	{\it Graduate Researcher}, UW--Madison
	%
		\begin{itemize}
		\item Applying machine learning methods to data collected using the IceCube Neutrino Observatory to study the cosmic-ray mass composition
		\end{itemize}

	\textsc{High Energy Phenomenology Group}  \hfill 2014-2015 \\
	{\it Graduate Researcher}, UW--Madison
	%
		\begin{itemize}
		\item Used effective field theory and simplified model methods to study dark matter signatures at collider experiments. In particular, searching for $Z'$ jets at the LHC.
		\end{itemize}


	\textsc{ATLAS Forward Proton  (AFP) Detector} \hfill 2010-2013 \\
	{\it Undergraduate Researcher}, UT--Arlington
	%
		\begin{itemize}
		\item Contributed to the development of the AFP detector system, a high-precision time-of-flight detector that was proposed as part of an upgrade to the ATLAS experiment at the LHC.
		%                 \begin{itemize}  \itemsep -2pt % reduce space between items
		%                 \renewcommand{\labelitemii}{$\circ$}
		%                 \item Evaluated the performance of microchannel plate (MCP) photomultiplier tubes (PMTs).
		%                 \item Wrote and maintained code base used to analyzed data collected in the lab (C++ and ROOT).
		%                 \end{itemize}
		\end{itemize}

	\textsc{Nanoparticle Scintillator Radiation Detection} \hfill 2010-2011 \\
	{\it Undergraduate Researcher}, UT--Arlington
	%
		\begin{itemize}
		\item Developed a radiation detection setup using PMTs and photodiodes to assess the performance of new nanoparticle scintillators.
		\end{itemize}

%============= Software =========================
\section{\sc Software}

	I am an active developer, maintainer, and contributor to several projects in the Python data science community. I'm the maintainer of:
	\begin{itemize}
		\item \href{https://github.com/jrbourbeau/pycondor}{PyCondor}--Python API for submitting tasks to an HTCondor distributed cluster.
		\item \href{https://github.com/WIPACRepo/decotools}{decotools}--Python package to help analyze data collected by the Distributed Electronic Cosmic-ray Observatory (DECO).
	\end{itemize}

	I've also made contributions to other open-source projects such as scikit-learn, dask, mlxtend, etc. See my \href{https://github.com/jrbourbeau}{GitHub profile} for full details.


%============= Talks =========================
\section{\sc Selected Talks}
		\textsc{Dark Matter Searches with a Mono-$Z'$ Jet} \hfill May 2015\\
		Phenomenology 2015 Symposium--Pittsburgh, PA\\ \\
		\textsc{Development of a Fast Timing System for the ATLAS \\ Forward Proton Detector} \hfill March 2012\\
		Contributed Talk. UTA Annual Celebration of Excellence by Students.
		  \begin{itemize}
                     \item Received the Provost's Award for an Undergraduate Oral Presentation.
                     \end{itemize}


%\section{\bf{Teaching Experience}}
%		\textsc{Graduate Teaching Assistant} \hfill 2013-2015 \\
%		{\it University of Wisconsin--Madison}
%		\\
%	         \begin{itemize}  \itemsep -2pt % reduce space between items
%	         \item Physics 201 (Calculus-based mechanics). \hfill Fall 2015
%	         \item Physics 104 (Algebra-based introductory E\&M). \hfill Spring 2015
%	         \item Physics 103 (Algebra-based introductory mechanics). \hfill Fall 2014
%	         \item Physics 104 (Algebra-based introductory E\&M). \hfill Spring 2014
%                  \item Physics 103 (Algebra-based introductory mechanics). \hfill Fall 2013
%	         \end{itemize}

%\section{\bf{Awards and Scholarships}}  \textsc{Undergraduate Research Assistantship} \hfill Summer 2012\\
%                  UTA Honors College\\
%                  \\
%                  \textsc{Bonnie Cecil and Jo Thompson Award} \hfill Spring 2012\\
%		UTA Physics Department\\
%                   \\
%                   %\textsc{Provost's Award for an Undergraduate Oral Presentation} \hfill Spring 2012\\
%                   %UTA Graduate School\\
%                   %\\
%                   \textsc{College of Science Dean's List} \hfill  2010-2012\\
%                   UT Arlington\\
%                   \\
%                    \textsc{UT Arlington Academic Achievement Scholarship \\ for Continuing Students} \hfill 2010-2013\\
%
%
%

\end{resume}
\end{document}
